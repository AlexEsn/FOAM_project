\documentclass{beamer}
\usetheme{Madrid}

\usepackage[T2A]{fontenc}
\usepackage[utf8]{inputenc}
\usepackage[main=russian,english]{babel}
\usepackage{graphicx}
\usepackage{enumitem}
\usepackage{biblatex}

\addbibresource{Overview.bib}

\begin{document}

\title{Исследование и оптимизация системы охлаждения}
\author{Выполнил: Есис А. И., Руководитель проекта: Чмыхов М. А.}
\maketitle

\section{Введение}

Цель данной работы заключается в исследовании и оптимизации системы охлаждения с использованием численных методов и оптимизации. В работе рассмотрены различные этапы исследования, включая анализ геометрии, оптимизацию процесса расчета, модификацию геометрии и параметров системы, применение эволюционных и градиентных методов оптимизации.

\section{Данные предыдущего этапа исследования}

На предыдущих этапах исследования была проведена оптимизация системы с использованием полного перебора и анализа геометрии. Многопоточность была использована для ускорения процесса расчета. Полученные результаты послужили отправной точкой для более глубокого исследования.

\section{Проблема с автоматической генерацией сетки в SALOME}

В процессе работы выявлена проблема с автоматической генерацией сетки в SALOME, а именно с ошибками в экспорте Python скрипта с сеткой. Рассмотрены проблемы и их решения, такие как SubShapeAllIDs, GetShapesOnBoxIDs, GetShapesOnPlaneWithLocationIDs.

\section{Автоматизированный процесс генерации сетки и расчета}

Был представлен автоматизированный процесс генерации сетки и расчета. Процесс включает в себя использование SALOME для построения геометрии, генерации сетки и расчета тепловых характеристик системы.

\section{Особенности фиксации и движения цилиндров}

Основное внимание уделено особенностям фиксации и движения цилиндров в системе. Рассмотрены различные подходы к этим аспектам и их влияние на тепловые процессы.

\section{Первые результаты}

На ранних этапах работы были получены первые результаты, включающие в себя зависимость максимальной температуры нагревателя от сдвигов цилиндров.

\begin{figure}[h]
	\centering
	\includegraphics[width=0.5\linewidth]{first_results.jpg}
	\caption{Пример зависимости температуры от сдвигов цилиндров.}
\end{figure}

\section{Оптимизация процесса расчета}

Для повышения эффективности анализа была проведена оптимизация процесса расчета с использованием многопоточности. Это позволило ускорить проведение исследований и сделать их более подробными.

\section{Увеличение количества точек для анализа}

С целью получения более детального исследования было увеличено количество точек для анализа. Эксперименты проводились с различными шагами, позволяя получить более точные представления о влиянии параметров на характеристики системы.

\section{Первые результаты и наблюдения}

Первые результаты анализа геометрии системы показали интересные закономерности. Обнаружено, что минимумы характеристик системы наблюдаются при равных сдвигах цилиндров, способствуя более эффективному теплообмену.

\section{Эксперименты с геометрией и анализ результатов}

Проведены эксперименты с геометрией системы, включая уменьшение диаметра цилиндров и изменение расположения. Получены более интересные результаты, подчеркивающие влияние геометрии на тепловые характеристики.

\section{Оптимизация расчетной сетки}

Для более глубокого анализа системы и выявления потенциальных выбросов в температурных данных была оптимизирована расчетная сетка, что позволило выделить особенности теплового распределения в системе.

\section{Модификация геометрии и уменьшение воздуховода}

Проведена серия изменений в геометрии системы, включая уменьшение воздуховода. Эти изменения привели к новым результатам и подчеркнули важность геометрических параметров для эффективности охлаждения.

\section{Анализ реальной системы и изменения параметров}

Был проведен анализ реальных условий системы охлаждения. Внесены изменения в скорость воздуха, материал радиатора и геометрию нагревателя, что привело к новым результатам и пониманию влияния параметров на тепловые характеристики.

\section{Применение эволюционного алгоритма в оптимизации}

Для более эффективного поиска оптимальных параметров в задаче охлаждения был применен эволюционный алгоритм. Это позволило достичь точек минимума всего за 50 расчетов, существенно сократив количество необходимых расчетов.

\section{Использование градиентного метода}

В работе был использован градиентный метод для оптимизации. Градиентные методы обеспечили быструю сходимость и точность результатов в задачах, где требуется высокая точность оптимизации.

\section{Заключение}

Работа предоставляет комплексный обзор исследования системы охлаждения, включая анализ геометрии, оптимизацию, модификацию параметров и применение различных методов оптимизации. Полученные результаты позволяют сделать вывод о важности оптимизации геометрии и параметров для повышения эффективности системы охлаждения.

\newpage
\printbibliography

\end{document}