\documentclass[a4paper,12pt]{article}

%%% Работа с русским языком
\usepackage{cmap}					% поиск в PDF
% \usepackage{mathtext} 				% русские буквы в формулах
\usepackage[T2A]{fontenc}			% кодировка
\usepackage[utf8]{inputenc}			% кодировка исходного текста
\usepackage[english,russian]{babel}	% локализация и переносы

%%% Страница
\usepackage{extsizes} % Возможность сделать 14-й шрифт
\usepackage{geometry} % Создание полей
\geometry{top = 2.5cm}
\geometry{bottom = 2cm}
\geometry{left = 3cm}
\geometry{right = 1.5cm}
\renewcommand{\baselinestretch}{1.5} % Интерлиньяж 1.5

%%% Дополнительная работа с математикой
\usepackage{amsmath,amsfonts,amssymb,amsthm,mathtools} % AMS
\usepackage{icomma} % "Умная" запятая: $0,2$ --- число, $0, 2$ --- перечисление

%% Номера формул
%\mathtoolsset{showonlyrefs=true} % Показывать номера только у тех формул, на которые есть \eqref{} в тексте.
%\usepackage{leqno} % Нумерация формул слева

%% Свои команды
\DeclareMathOperator{\sgn}{\mathop{sgn}}

%% Перенос знаков в формулах (по Львовскому)
\newcommand*{\hm}[1]{#1\nobreak\discretionary{}
	{\hbox{$\mathsurround=0pt #1$}}{}}

%%% Работа с картинками
\usepackage{graphicx}  % Для вставки рисунков
\graphicspath{{images/}{images2/}}  % папки с картинками
\setlength\fboxsep{3pt} % Отступ рамки \fbox{} от рисунка
\setlength\fboxrule{1pt} % Толщина линий рамки \fbox{}
\usepackage{wrapfig} % Обтекание рисунков текстом

%%% Работа с таблицами
\usepackage{array,tabularx,tabulary,booktabs} % Дополнительная работа с таблицами
\usepackage{longtable}  % Длинные таблицы
\usepackage{multirow} % Слияние строк в таблице

%%% Теоремы
\theoremstyle{plain} % Стиль по умолчанию
\newtheorem{theorem}{Теорема}[section]
\newtheorem{proposition}[theorem]{Утверждение}

\theoremstyle{definition} % "Определение"
\newtheorem{corollary}{Следствие}[theorem]
\newtheorem{problem}{Задача}[section]

\theoremstyle{remark} % "Примечание"
\newtheorem*{nonum}{Решение}

%%% Программирование
\usepackage{etoolbox} % логические операторы

\usepackage{lastpage} % Узнать, сколько всего страниц в документе.

\usepackage{soul} % Модификаторы начертания

\usepackage{graphicx} %Подключаю модули для добавления картинок
\graphicspath{.}
\DeclareGraphicsExtensions{.pdf,.png,.jpg}
	
	\renewcommand{\baselinestretch}{1.5}
	
\begin{document}
\renewcommand{\contentsname}{\Large Содержание}
\renewcommand{\bibname}{\normalfont\Large\bfseries Список литературы}

\begin{titlepage}
	\begin{center}
		Министерство науки и высшего образования Российской Федерации \\
		НАЦИОНАЛЬНЫЙ ИССЛЕДОВАТЕЛЬСКИЙ ЯДЕРНЫЙ УНИВЕРСИТЕТ <<МИФИ>> \\*
		\hrulefill
	\end{center}
	
	\begin{center}
		ИНСТИТУТ ЛАЗЕРНЫХ И ПЛАЗМЕННЫХ ТЕХНОЛОГИЙ\\
		КАФЕДРА №31 ПРИКЛАДНАЯ МАТЕМАТИКА
	\end{center}
	\vspace{1cm}
	
	\vspace{2em}
	
	\begin{center}
		\large{Отчет}
		
		по научно-исследовательской работе на тему:
	\end{center}
	\begin{center}
		\large <<Оптимизация канального радиатора>>
	\end{center}
	\begin{center}
		\large \textit{Выполнил: Есис А. И.}
		
		\textit{Руководитель проекта: Чмыхов М. А.}
	\end{center}
	
	
	\vspace{22em}
	
	\begin{center}
		г. Москва 2023
	\end{center}
\end{titlepage}

\newpage
\section*{Аннотация}

Данная работа посвящена исследованию и оптимизации формы радиатора с применением трех программных инструментов: OpenFOAM, ParaView и SALOME Meca. Целью исследования является повышение эффективности радиатора путем оптимизации его геометрии.

В начале исследования используется SALOME Meca для построения геометрии и сетки радиатора. На втором этапе проводятся численные симуляции с использованием OpenFOAM. В результате симуляций получаются данные о тепловом и газодинамическом поведении радиатора. Наконец, результаты численных симуляций визуализируются с помощью ParaView. И делается вывод о наиболее оптимальной форме радиатора.

\newpage 
\tableofcontents
\setcounter{page}{3}

\newpage
\section{Введение}


OpenFOAM, ParaView и SALOME Meca являются мощными программными инструментами, широко применяемыми в области вычислительной гидрогазодинамики (CFD) и численного моделирования. Вместе они предоставляют комплексное решение для анализа и визуализации сложных физических процессов, таких как течение жидкостей, теплообмен, движение твердых тел и другие.

OpenFOAM (Open Field Operation and Manipulation) является свободным и открытым программным обеспечением для решения уравнений Навье-Стокса и других математических моделей, связанных с течением жидкостей и газов. OpenFOAM предоставляет широкий спектр методов решения, таких как конечно-разностные, конечно-объемные и конечно-элементные, что позволяет исследовать различные типы потоков и применять разные физические модели. Будучи свободно распространяемым и расширяемым, OpenFOAM предоставляет возможность настраивать и адаптировать код под конкретные задачи и требования.

ParaView является инструментом, который позволяет визуализировать результаты численных симуляций. Он предоставляет широкий спектр функций для создания визуализаций, включая 2D и 3D графику, построение контуров, срезов и анимаций. ParaView также поддерживает интерактивное взаимодействие с моделью, что позволяет анализировать данные в реальном времени, изменять параметры симуляции и осуществлять глубокий анализ результатов.

SALOME Meca является интегрированной средой для предварительной обработки геометрии и настройки расчетной сетки для OpenFOAM. Он предоставляет интуитивный пользовательский интерфейс, который облегчает создание и манипулирование сложными геометрическими моделями, а также настройку сеток с различной структурой. SALOME Meca также предлагает набор инструментов для проверки качества сетки и подготовки ее к последующим симуляциям в OpenFOAM \cite{wOfDocSalome}.

Вместе OpenFOAM, ParaView и SALOME Meca образуют мощный комплект инструментов для моделирования и визуализации в области вычислительной гидрогазодинамики. Они предоставляют возможность проводить сложные численные симуляции, анализировать результаты и визуализировать данные, что помогает в понимании физических процессов и принятии информированных решений в различных областях, таких как авиация, автомобильная промышленность, энергетика и многое другое.

\newpage
\section{предыдущий семестр}

В данном проекте исследуется задача охлаждения нагретого тела с помощью установки радиатора.

В начале проекта геометрия радиатора строится с использованием графического интерфейса SALOME. Используя интуитивно понятный пользовательский интерфейс, создается геометрия радиатора. Этот этап позволяет получить исходную геометрию, которая будет использоваться для последующих анализов.

Затем, для повышения гибкости и автоматизации процесса, создается скрипт на языке Python, который генерирует геометрию радиатора. В этом скрипте можно устанавливать параметры радиатора, такие как расположение элементов. Это позволяет быстро создавать и изменять различные варианты геометрии радиатора для дальнейшего анализа и оптимизации.

Для исследования были использованы три различных варианта геометрии радиатора.
Примеры геометрии:
\begin{figure}[h]
	\begin{center}
		\includegraphics[width=0.5\linewidth]{1.1.png}
		\caption{Модель геометрии 1} %% подпись к рисунку
	\end{center}
\end{figure}
\begin{figure}[h]
	\begin{center}
		\includegraphics[width=0.5\linewidth]{1.2.png}
		\caption{Модель геометрии 2}
	\end{center}
\end{figure}
\begin{figure}[h]
	\begin{center}
		\includegraphics[width=0.5\linewidth]{1.3.png}
		\caption{Модель геометрии 3}
	\end{center}
\end{figure}
\newpage

\begin{figure}[h]
	\begin{center}
		\includegraphics[width=0.5\linewidth]{2.1.png}
		\caption{Модель геометрии 1} %% подпись к рисунку
	\end{center}
\end{figure}
\begin{figure}[h]
	\begin{center}
		\includegraphics[width=0.5\linewidth]{2.2.png}
		\caption{Модель геометрии 2}
	\end{center}
\end{figure}
\newpage
\begin{figure}[h]
	\begin{center}
		\includegraphics[width=0.5\linewidth]{2.3.png}
		\caption{Модель геометрии 3}
	\end{center}
\end{figure}

\par
В данных моделях части радиатора (3 цилиндрических элемента) являются подвижными и могут быть сдвинуты по оси Ox. Их радиус основания равен 5 мм, а высота 10 мм. 
Размеры подложки радиатора по ширине и длине 30 мм, а по высоте 2 мм. Нагреватель представляет собой куб с длиной ребра 10 мм и объемной плотностью источников тепла 2.94 * $10^7$ Вт/$м^3$. 
Скорость потока воздуха 5.6 м/с. Воздух находится в параллелепипеде размерами 300 мм по длине, 50 мм по ширине и 100 мм по высоте. Начальная температура –296.9 К. Параметры материалов приведены в таблице 1 \cite{aHeatTranserf}.

\begin{table}[h]
	\begin{tabular}{|l|l|l|l|}
		\hline
		                                          & Нагреватель & Радиатор & Воздух        \\
		Плотность {[}кг/$м^3${]}                  & 1280        & 2700     & 1.196         \\
		\hline
		Cp   {[}Дж/кг*К{]}                        & 1004        & 900      & 1005          \\
		\hline
		Коэффициент теплопроводности {[}Вт/м*К{]} & 80          & 200      &               \\
		\hline
		Молекулярная масса {[}г/моль{]}           & 50          & 27       & 28.9          \\
		\hline
		Вязкость {[}кг/м*с{]}                     &             &          & $1.8*10^{-5}$ \\
		\hline
		Число Прандтля                            &             &          & 0.7           \\
		\hline
	\end{tabular}
	\caption{Параметры задачи} %% подпись к рисунку
\end{table}

Затем по модели была построена сетка c уплотнением в области радиатора и нагревателя и произведено разбиение на регионы:

\begin{figure}[h]
	\begin{center}
		\includegraphics[width=0.35\linewidth]{3.png}
		\caption{Сетка модели 1} %% подпись к рисунку
	\end{center}
\end{figure}

\par
Для численного решения задачи используется решатель chtMultiRegionFoam.
Он применяется для расчета теплообмена между жидкостью/газом и твердым телом. А также для моделирования сложных задач, связанных с теплопередачей и теплообменом в многорегиональных системах \cite{wChtMultiRegionFoam}.
Для каждого региона задавались начальные и граничные условия. Были также заданы дополнительные функции для анализа \cite{aHeatTranserf}.
Для моделей были получены следующие результаты распределений по температурам:

\begin{figure}[h]
	\begin{center}
		\includegraphics[width=0.45\linewidth]{5.1.png}
		\caption{Распределение температуры для модели 1} %% подпись к рисунку
	\end{center}
\end{figure}

\begin{figure}[h]
	\begin{center}
		\includegraphics[width=0.45\linewidth]{5.2.png}
		\caption{Распределение температуры для модели 2} %% подпись к рисунку
	\end{center}
\end{figure}

\begin{figure}[h]
	\begin{center}
		\includegraphics[width=0.45\linewidth]{5.3.png}
		\caption{Распределение температуры для модели 3} %% подпись к рисунку
	\end{center}
\end{figure}

\newpage
Были получены следующие результаты для моделей:
\begin{enumerate}
	\item \textbf{Первая модель:}
	      \begin{itemize}
		      \item Средняя температура радиатора: 508.381 К
		      \item Средняя температура нагревателя: 533.363 К
		      \item Средняя температура интерфейса между нагревателем и радиатором: 521.537 К
	      \end{itemize}
	\item \textbf{Вторая модель:}
	      \begin{itemize}
		      \item Средняя температура радиатора: 555.23 К
		      \item Средняя температура нагревателя: 576.306 К
		      \item Средняя температура интерфейса между нагревателем и радиатором: 564.451 К
	      \end{itemize}
	\item \textbf{Третья модель:}
	      \begin{itemize}
		      \item Средняя температура радиатора: 519.325 К
		      \item Средняя температура нагревателя: 537.741 К
		      \item Средняя температура интерфейса между нагревателем и радиатором: 525.862 К
	      \end{itemize}
\end{enumerate}

Построены графики температуры в нагревателе:
\newpage
\begin{figure}[h]
	\begin{center}
		\includegraphics[width=0.4\linewidth]{6.1.png}
		\caption{Распределение температуры в нагревателе для модели 1} %% подпись к рисунку
	\end{center}
\end{figure}
\begin{figure}[h]
	\begin{center}
		\includegraphics[width=0.4\linewidth]{6.2.png}
		\caption{Распределение температуры в нагревателе для модели 2} %% подпись к рисунку
	\end{center}
\end{figure}
\newpage
\begin{figure}[h]
	\begin{center}
		\includegraphics[width=0.4\linewidth]{6.3.png}
		\caption{Распределение температуры в нагревателе для модели 3} %% подпись к рисунку
	\end{center}
\end{figure}
\begin{figure}[h]
	\begin{center}
		\includegraphics[width=0.4\linewidth]{6.png}
		\caption{Распределение температур в нагревателях для всех моделей} %% подпись к рисунку
	\end{center}
\end{figure}

\newpage
\par
Затем было проведено 2000 итераций расчета. Для отслеживания сходимости во время проведения расчета были построены следующие графики (пример для модели 1):
\begin{figure}[h]
	\begin{center}
		\includegraphics[width=0.35\linewidth]{11.1.png}
		\caption{Зависимость максимальной температуры радиатора от шага} %% подпись к рисунку
	\end{center}
\end{figure}
\begin{figure}[h]
	\begin{center}
		\includegraphics[width=0.35\linewidth]{12.1.png}
		\caption{Зависимость средней температуры радиатора от шага} %% подпись к рисунку
	\end{center}
\end{figure}
\begin{figure}[h]
	\begin{center}
		\includegraphics[width=0.35\linewidth]{13.1.png}
		\caption{Зависимость средней температуры интерфейса между нагревателем и радиатором от шага} %% подпись к рисунку
	\end{center}
\end{figure}
\newpage
\begin{figure}[h]
	\begin{center}
		\includegraphics[width=0.4\linewidth]{14.1.png}
		\caption{Зависимость максимальной температуры нагревателя от шага} %% подпись к рисунку
	\end{center}
\end{figure}
\begin{figure}[h]
	\begin{center}
		\includegraphics[width=0.4\linewidth]{15.1.png}
		\caption{Зависимость средней температуры нагревателя от шага} %% подпись к рисунку
	\end{center}
\end{figure}

\par
Далее были построены графики распределений температуры в центральном сечении:

\begin{figure}[h]
	\begin{center}
		\includegraphics[width=0.4\linewidth]{7.1.png}
		\caption{Распределение температуры для модели 1} %% подпись к рисунку
	\end{center}
\end{figure}

\begin{figure}[h]
	\begin{center}
		\includegraphics[width=0.4\linewidth]{7.2.png}
		\caption{Распределение температуры для модели 2} %% подпись к рисунку
	\end{center}
\end{figure}
\newpage
\begin{figure}[h]
	\begin{center}
		\includegraphics[width=0.4\linewidth]{7.3.png}
		\caption{Распределение температуры для модели 3} %% подпись к рисунку
	\end{center}
\end{figure}

\par
И графики распределений скоростей:
\begin{figure}[h]
	\begin{center}
		\includegraphics[width=0.4\linewidth]{8.1.png}
		\caption{Распределение скоростей для модели 1} %% подпись к рисунку
	\end{center}
\end{figure}

\begin{figure}[h]
	\begin{center}
		\includegraphics[width=0.4\linewidth]{8.2.png}
		\caption{Распределение скоростей для модели 2} %% подпись к рисунку
	\end{center}
\end{figure}
\newpage
\begin{figure}[h]
	\begin{center}
		\includegraphics[width=0.4\linewidth]{8.3.png}
		\caption{Распределение скоростей для модели 3} %% подпись к рисунку
	\end{center}
\end{figure}
\begin{figure}[h]
	\begin{center}
		\includegraphics[width=0.4\linewidth]{9.png}
		\caption{Распределение скоростей для модели 1} %% подпись к рисунку
	\end{center}
\end{figure}

\section{Проблема с Генерацией Сетки в SALOME}

В ходе нашего исследования мы столкнулись с проблемой при перегенерации геометрии в SALOME, используя скрипты Python API. Проблема заключалась в том, что при каждой попытке обновления геометрии с новыми данными, мы получали сетку с ошибками из-за нарушения идентификаторов (ID), так как они изменялись.

Для решения этой проблемы мы обратились к Python API в SALOME и использовали следующие методы:

\subsection{\texttt{SubShapeAllIDs}}

Функция \texttt{SubShapeAllIDs} в Python API SALOME используется для получения всех идентификаторов (ID) подформ в геометрии. Это полезно, когда необходимо провести операции с каждой подформой в модели.

Пример использования:
\begin{verbatim}
subshape_ids = geompy.SubShapeAllIDs(main_shape)
print(f"ID подформ: {subshape_ids}")
\end{verbatim}

В данном примере \texttt{main\_shape} - это главная форма в вашей геометрии.

\subsection{\texttt{GetShapesOnBoxIDs}}

Функция \texttt{GetShapesOnBoxIDs} возвращает идентификаторы форм, которые содержатся внутри заданного объема, определенного прямоугольным параллелепипедом.

Пример использования:
\begin{verbatim}
box = geompy.MakeBox(0, 0, 0, 10, 10, 10)
shapes_inside_box_ids = geompy.GetShapesOnBoxIDs(main_shape, box)
print(f"ID форм внутри прямоугольного параллелепипеда: {shapes_inside_box_ids}")
\end{verbatim}

Здесь \texttt{main\_shape} - это опять же главная форма, а \texttt{box} - созданный прямоугольный параллелепипед.

\subsection{\texttt{GetShapesOnPlaneWithLocationIDs}}

Функция \texttt{GetShapesOnPlaneWithLocationIDs} возвращает идентификаторы форм, которые пересекают заданную плоскость.

Пример использования:
\begin{verbatim}
plane = geompy.MakePlane(0, 0, 1, 0)
shapes_on_plane_ids = geompy.GetShapesOnPlaneWithLocationIDs(main_shape, plane)
print(f"ID форм, пересекаемых плоскостью: {shapes_on_plane_ids}")
\end{verbatim}

В этом примере \texttt{main\_shape} - ваша главная форма, а \texttt{plane} - созданная плоскость.

Используя эти методы, мы смогли стабилизировать процесс генерации сетки, обеспечивая постоянство идентификаторов форм даже при обновлении геометрии. Это решение позволило нам успешно использовать скрипты Python API для перегенерации геометрии в дальнейших этапах нашего исследования.

\section{Автоматизированный процесс генерации сетки и расчета}

В данном разделе рассматривается процесс автоматизации генерации сетки в программе SALOME и выполнения расчетов в OpenFOAM для различных комбинаций параметров. Ключевыми параметрами, подлежащими варьированию, являются \texttt{shift\_first\_cylinder} и \texttt{shift\_second\_cylinder}.

Переменная \texttt{shift\_first\_cylinder} представляет собой сдвиг первого цилиндра в радиаторе, а \texttt{shift\_second\_cylinder} — сдвиг второго цилиндра. Эти параметры влияют на геометрию радиатора и, следовательно, на условия теплообмена в системе.

Скрипт создает уникальное имя для каждой комбинации параметров и затем копирует исходный кейс в новую директорию с уникальным именем. После этого SALOME запускается в режиме командной строки для выполнения генерации сетки с учетом новой геометрии, заданной параметрами сдвига цилиндров. Далее запускаются соответствующие bash-скрипты для выполнения расчета сетки и кейса в OpenFOAM.

После завершения расчета, автоматически извлекаются результаты. Сценарий ищет и анализирует файл с данными теплообмена в постпроцессинговой директории. Это позволяет собирать и систематизировать конечные результаты для каждой конфигурации геометрии. Такой подход обеспечивает эффективный анализ влияния различных параметров на теплоотдачу системы.

\section{Особенности Фиксации и Движения Цилиндров}

Необходимо отметить, что в проведенных исследованиях у нас было три цилиндра в системе. Тем не менее, третий цилиндр был зафиксирован в положении (0, 0), и двигались только два оставшихся цилиндра. Такой выбор сделан с целью упростить визуализацию и анализ результатов.

Такой подход облегчает анализ и интерпретацию результатов, обеспечивая более ясное представление о влиянии конкретных параметров на эффективность теплообмена в рассматриваемой системе.

\section{Первые Результаты и Интерполяция Поверхности}

На основе автоматизированного процесса были получены первые результаты, охватывающие 25 точек варьирования параметров. Исследуемые параметры \texttt{shift\_first\_cylinder} и \texttt{shift\_second\_cylinder} изменялись от 0 до 20 с шагом 5. Эти значения представляют различные комбинации сдвигов цилиндров в радиаторе, что отражает влияние геометрических параметров на теплоотдачу системы.

\begin{figure}[h]
	\begin{center}
		\includegraphics[width=0.4\linewidth]{16.1.jpg}
		\caption{Зависимость максимальной температуры нагревателя от сдвигов цилиндров} %% подпись к рисунку
	\end{center}
\end{figure}

\begin{figure}[h]
	\begin{center}
		\includegraphics[width=0.4\linewidth]{16.2.jpg}
		\caption{Зависимость максимальной температуры нагревателя от сдвигов цилиндров} %% подпись к рисунку
	\end{center}
\end{figure}
\newpage
\begin{figure}[h]
	\begin{center}
		\includegraphics[width=0.4\linewidth]{16.3.jpg}
		\caption{Зависимость максимальной температуры нагревателя от сдвигов цилиндров} %% подпись к рисунку
	\end{center}
\end{figure}

Для улучшения визуализации и наглядности полученных данных была проведена интерполяция поверхности. Интерполированная поверхность дает представление о поведении системы в пространстве параметров и позволяет выявить области оптимальных значений для исследуемых параметров.

Этот этап анализа предоставляет первичное представление о влиянии сдвига цилиндров на теплообмен в системе.

\section{Оптимизация процесса расчета}

С целью оптимизации процесса расчета для большего количества точек был реализован многопоточный код на языке программирования Python. Для этого был использован модуль multiprocessing, который позволяет создавать и управлять параллельными процессами.

Прежде всего, были внесены некоторые изменения в существующий код. Введены параметры \texttt{shift\_third\_cylinder} и \texttt{all\_data} для адаптации к третьему цилиндру и сбору результатов в разделяемом словаре. Это позволило более гибко настраивать геометрию и эффективно собирать результаты расчетов.

С использованием модуля multiprocessing была реализована функция \texttt{run\_simulation}, которая выполняет расчет для каждой комбинации параметров в отдельном процессе. Результаты каждого процесса сохраняются в разделяемом словаре \texttt{all\_data}, где ключами являются уникальные идентификаторы (например, \texttt{\_0\_0\_0}), а значениями — результаты расчетов.

Затем, с использованием библиотеки Pool и метода \texttt{starmap}, были запущены отдельные процессы для каждой комбинации параметров. Это позволило распараллелить выполнение расчетов и значительно ускорить процесс.

Результаты расчетов были собраны в разделяемом словаре \texttt{all\_data}, который после завершения всех процессов может быть использован для анализа и построения интерполяции. Это позволяет обработать большой объем данных более эффективно, особенно при увеличении числа комбинаций параметров для анализа.

\section{Увеличение количества точек для анализа}

Оптимизация процесса расчета с использованием многопоточности значительно повысила эффективность анализа.

Так, на следующем этапе было проведено исследование с шагом 2, что дало 121 точку для анализа. Это уже более подробное и детальное исследование, что позволяет получить более точные представления о влиянии параметров на характеристики системы.

\begin{figure}[h]
	\begin{center}
		\includegraphics[width=0.4\linewidth]{17.1.jpg}
		\caption{Зависимость максимальной температуры нагревателя от сдвигов цилиндров (121 точка)} %% подпись к рисунку
	\end{center}
\end{figure}
\begin{figure}[h]
	\begin{center}
		\includegraphics[width=0.4\linewidth]{17.2.jpg}
		\caption{Зависимость максимальной температуры нагревателя от сдвигов цилиндров (121 точка)} %% подпись к рисунку
	\end{center}
\end{figure}

Затем мы перешли к шагу 1, что привело к анализу 441 точки.

\begin{figure}[h]
	\begin{center}
		\includegraphics[width=0.4\linewidth]{18.1.jpg}
		\caption{Зависимость максимальной температуры нагревателя от сдвигов цилиндров (441 точка)} %% подпись к рисунку
	\end{center}
\end{figure}
\begin{figure}[h]
	\begin{center}
		\includegraphics[width=0.4\linewidth]{18.2.jpg}
		\caption{Зависимость максимальной температуры нагревателя от сдвигов цилиндров (441 точка)} %% подпись к рисунку
	\end{center}
\end{figure}
\begin{figure}[h]
	\begin{center}
		\includegraphics[width=0.4\linewidth]{18.3.jpg}
		\caption{Зависимость максимальной температуры нагревателя от сдвигов цилиндров (441 точка)} %% подпись к рисунку
	\end{center}
\end{figure}

\section{Первые результаты и наблюдения}

Первые результаты анализа геометрии системы показали интересные закономерности. В частности, было отмечено, что минимумы характеристик системы наблюдаются при равных сдвигах первого и второго цилиндров. Это явление подтверждается не только экспериментально, но и с теоретической точки зрения.

Анализируя геометрию, можно отметить, что при равных сдвигах цилиндров они касаются друг друга. Такая конфигурация способствует более эффективному теплообмену между цилиндрами и более равномерному распределению тепловой нагрузки. Касание цилиндров создает условия для оптимального теплопередачи и распределения тепла, что приводит к минимизации температурных градиентов и, следовательно, к минимуму искомых характеристик.

\begin{figure}[h]
	\begin{center}
		\includegraphics[width=0.4\linewidth]{19.jpg}
		\caption{Геометрия при которой достигается минимальное значение} %% подпись к рисунку
	\end{center}
\end{figure}

\section{Эксперименты с геометрией}

Однако эта геометрия не демонстрировала реальной зависимости между расположением цилиндров и температурой. В ответ на это наблюдение был проведен ряд экспериментов с общей геометрией системы.

В первом эксперименте было решено уменьшить диаметр цилиндров в два раза, сделав их тоньше (2.5 мм). Это привело к тому, что цилиндры перестали касаться друг друга, создавая новые условия для теплообмена. Теперь теплообмен между цилиндрами осуществляется через воздушный зазор.

\begin{figure}[h]
	\begin{center}
		\includegraphics[width=0.4\linewidth]{20.jpg}
		\caption{Новая геометрия} %% подпись к рисунку
	\end{center}
\end{figure}


\section{Изменение геометрии и результаты}

После внесения изменений в геометрию системы были получены более интересные результаты, которые отличаются от предыдущих экспериментов. Новая конфигурация цилиндров более чувствительна к расположению внутри системы, и температурные различия стали более заметными.

Например, можно заметить, что при размещении ближе к центру системы температуры становятся меньше. Это может быть обусловлено более равномерным распределением тепла в системе и более эффективным теплообменом.

\begin{figure}[h]
	\begin{center}
		\includegraphics[width=0.4\linewidth]{21.1.jpg}
		\caption{Зависимость максимальной температуры нагревателя от сдвигов цилиндров} %% подпись к рисунку
	\end{center}
\end{figure}
\begin{figure}[h]
	\begin{center}
		\includegraphics[width=0.4\linewidth]{21.2.jpg}
		\caption{Зависимость максимальной температуры нагревателя от сдвигов цилиндров} %% подпись к рисунку
	\end{center}
\end{figure}
\begin{figure}[h]
	\begin{center}
		\includegraphics[width=0.4\linewidth]{21.3.jpg}
		\caption{Зависимость максимальной температуры нагревателя от сдвигов цилиндров} %% подпись к рисунку
	\end{center}
\end{figure}

В ходе экспериментов было обнаружено, что существует оптимальная конфигурация, при которой достигается минимум температуры в системе. Эта конфигурация характеризуется расстановкой цилиндров по диагонали.

\begin{figure}[h]
	\begin{center}
		\includegraphics[width=0.4\linewidth]{21.4.jpg}
		\caption{Пример миниму при новой геометрии} %% подпись к рисунку
	\end{center}
\end{figure}

А также при данной конфигурации будет достигаться минимум максимальной температуры нагревателя.

\begin{figure}[h]
	\begin{center}
		\includegraphics[width=0.4\linewidth]{21.5.jpg}
		\caption{Пример минимума при новой геометрии} %% подпись к рисунку
	\end{center}
\end{figure}

\section{Оптимизация расчетной сетки}

Для более глубокого анализа системы и выявления потенциальных выбросов в температурных данных, мы решили оптимизировать расчетную сетку. Используя параметры NETGEN_3D, мы уменьшили размер элементов сетки.

Этот подход позволил нам более детально рассмотреть поведение системы в местах с высокими градиентами температуры и выделить потенциальные выбросы. Результаты показали, что некоторые точки действительно являются выбросами, что может быть связано с особенностями теплового распределения в этих областях.

\section{Модификация геометрии и уменьшение воздуховода}

Далее мы провели серию изменений в геометрии системы. Одним из ключевых моментов было уменьшение воздуховода практически до минимального зазора в 2 мм к цилиндрам. Это решение позволило избежать дополнительных условий на границе между воздуховодом и цилиндрами, создавая более естественные условия для теплового обмена.

С учетом внесенных изменений в геометрию системы и уменьшения воздуховода, мы провели новый ряд расчетов и получили следующие результаты.

\begin{figure}[h]
	\begin{center}
		\includegraphics[width=0.4\linewidth]{22.1.jpg}
		\caption{Зависимость максимальной температуры нагревателя от сдвигов цилиндров} %% подпись к рисунку
	\end{center}
\end{figure}
\begin{figure}[h]
	\begin{center}
		\includegraphics[width=0.4\linewidth]{22.2.jpg}
		\caption{Зависимость максимальной температуры нагревателя от сдвигов цилиндров} %% подпись к рисунку
	\end{center}
\end{figure}
\begin{figure}[h]
	\begin{center}
		\includegraphics[width=0.4\linewidth]{22.3.jpg}
		\caption{Скорость течения} %% подпись к рисунку
	\end{center}
\end{figure}
\begin{figure}[h]
	\begin{center}
		\includegraphics[width=0.4\linewidth]{22.4.jpg}
		\caption{Новая геометрия} %% подпись к рисунку
	\end{center}
\end{figure}

\section{Анализ реальной системы и изменения параметров}

Далее в нашем исследовании мы перешли к анализу более реалистичных условий, представляющих реальную систему охлаждения. Внесенные изменения включают снижение скорости воздуха внутри воздуховода до 3 м/с (по сравнению с предыдущим значением 5.6 м/с), замену материала радиатора на медь (предыдущий материал - алюминий), а также модификацию геометрии нагревателя, сделав его тоньше и немного меньше по размерам, а саму подложку радиатора сделав толще.

\begin{figure}[h]
	\begin{center}
		\includegraphics[width=0.4\linewidth]{23.1.jpg}
		\caption{Измененная геометрия} %% подпись к рисунку
	\end{center}
\end{figure}
\begin{figure}[h]
	\begin{center}
		\includegraphics[width=0.4\linewidth]{23.2.jpg}
		\caption{Зависимость максимальной температуры нагревателя от сдвигов цилиндров} %% подпись к рисунку
	\end{center}
\end{figure}
\begin{figure}[h]
	\begin{center}
		\includegraphics[width=0.4\linewidth]{23.3.jpg}
		\caption{Зависимость максимальной температуры нагревателя от сдвигов цилиндров} %% подпись к рисунку
	\end{center}
\end{figure}

\section{Применение эволюционного алгоритма в оптимизации}

Для более эффективного и быстрого поиска оптимальных параметров в задаче охлаждения был применен эволюционный алгоритм. Эволюционные алгоритмы — это класс методов оптимизации, вдохновленных процессами биологической эволюции. Они включают в себя механизмы отбора, скрещивания и мутации, а применительно к задачам оптимизации, такие алгоритмы позволяют находить оптимальные решения в пространствах больших размерностей.

Принцип работы эволюционного алгоритма можно кратко описать следующим образом:
1. **Инициализация популяции:** Создается начальная популяция индивидов (наборов параметров) случайным образом или на основе каких-то эвристик.

2. **Оценка приспособленности:** Каждый индивид из популяции оценивается по степени приспособленности в соответствии с целевой функцией. В нашем контексте целевой функцией могла бы быть, например, минимизация температурного поля внутри системы.

3. **Отбор:** Выбираются наиболее приспособленные индивиды для следующего поколения. Это может происходить различными методами, такими как турнирный отбор, рулеточный отбор и др.

4. **Скрещивание:** Происходит кроссовер (скрещивание) между выбранными индивидами, что приводит к созданию новых индивидов. Это позволяет объединить положительные черты родителей.

5. **Мутация:** Некоторые индивиды могут подвергаться мутациям, изменяя свои параметры с определенной вероятностью. Это вносит элемент случайности и разнообразия в популяцию.

6. **Повторение:** Описанные шаги повторяются в цикле до достижения критерия остановки, такого как заданное количество поколений или достижение требуемой точности.

Эволюционные алгоритмы являются мощным инструментом для оптимизации в больших пространствах параметров, позволяя находить приближенно оптимальные решения в условиях ограниченной информации о системе.

Применение эволюционного алгоритма в задаче оптимизации параметров системы охлаждения значительно снизило количество необходимых расчетов. Вместо полного перебора всех 441 точек в пространстве параметров, алгоритм смог достичь точек минимума всего за 50 расчетов.

\section{Использование градиентного метода}

Помимо дифференциальной эволюции, наша работа включала использование градиентного метода для оптимизации. Градиентные методы основаны на использовании градиента (производной) целевой функции для нахождения экстремума. В контексте оптимизации, где целью является минимизация или максимизация функции, градиентный метод может эффективно приближаться к оптимальному решению.

В рамках градиентного метода особенно полезными являются методы оптимизации с использованием градиента первого и второго порядка, такие как метод наименьших квадратов (L-BFGS-B) или метод сопряженных градиентов (CG).

Преимущества использования градиентных методов включают:

1. **Быстрая сходимость**: Градиентные методы могут сходиться быстро, особенно при правильном выборе параметров и хорошей обусловленности задачи.

2. **Эффективность на гладких функциях**: Если функция, которую мы оптимизируем, гладкая и дифференцируема, градиентные методы часто оказываются эффективными.

3. **Точность результатов**: Градиентные методы могут предоставить точные результаты в задачах, где необходимо достичь высокой точности оптимизации.

\begin{figure}[h]
	\begin{center}
		\includegraphics[width=0.4\linewidth]{24.jpg}
		\caption{Зависимость максимальной температуры нагревателя от сдвигов цилиндров} %% подпись к рисунку
	\end{center}
\end{figure}


\newpage
\section{Заключение}

В ходе нашего исследования мы провели комплексное исследование системы охлаждения, используя численные методы и оптимизацию. Ниже представлены ключевые выводы и результаты нашей работы:

1. **Геометрические изменения**: Изначально мы провели анализ геометрии системы, исследуя влияние расположения цилиндров на эффективность охлаждения. Эксперименты с различными конфигурациями позволили нам выявить оптимальные расстановки и влияние контакта цилиндров друг с другом на тепловые характеристики.

2. **Оптимизация с использованием полного перебора**: Мы провели первоначальную оптимизацию системы, используя полный перебор, что позволило нам выявить оптимальные точки в пространстве параметров. Это дало общий обзор зависимостей и минимумов целевой функции.

3. **Многопоточность для улучшения производительности**: В процессе увеличения количества точек в пространстве параметров мы использовали многопоточность для оптимизации, что значительно ускорило процесс расчета и позволило нам провести более подробное исследование.

4. **Эволюционный алгоритм**: Мы применили эволюционный алгоритм для оптимизации системы. Этот метод позволил нам более эффективно находить точки минимума, снижая количество расчетов по сравнению с полным перебором.

5. **Градиентные методы оптимизации**: Использование градиентных методов, таких как L-BFGS-B, добавило эффективность и точность в наш арсенал оптимизационных методов.

6. **Изменения в геометрии и параметрах системы**: Мы исследовали влияние различных параметров, таких как скорость воздуха и материалы, на эффективность охлаждения. Изменения в геометрии и параметрах системы привели к новым интересным результатам, позволяя лучше понять зависимости и оптимизировать систему.

7. **Использование алгоритмов оптимизации для более сложных сценариев**: Мы применили эти методы для более сложных сценариев, таких как изменение геометрии и параметров системы. Оптимизация позволила нам находить оптимальные конфигурации в условиях ограниченных вычислительных ресурсов.

8. **Выводы и перспективы**: На основе результатов исследования мы делаем вывод о том, что оптимизация геометрии и параметров системы охлаждения может существенно повысить ее эффективность. Подход с использованием различных методов оптимизации и численного моделирования предоставляет мощный инструментарий для разработки и улучшения теплоотводящих систем. В будущем возможно проведение более глубоких исследований с учетом дополнительных факторов и условий эксплуатации системы.

\newpage
\bibliographystyle{utf8gost705u}  %% стилевой файл для оформления по ГОСТу
\bibliography{Overview}     %% имя библиографической базы (bib-файла) 

\newpage
\section{Приложение А}
% \renewcommand{\thesection}{\Asbuk{section}}
Проект можно найти на github (github.com/AlexEsn/FOAM\_project), где содержатся все скрипты и кейсы.

\newpage
\section{Приложение Б}
% \renewcommand{\thesection}{\Asbuk{section}}
В процессе работы был найден баг в SALOME-9.9.0: при дампе Python скрипта с сеткой создается файл с ошибкой.

\begin{figure}[h]
	\begin{center}
		\includegraphics[width=1\linewidth]{e.png}
		\caption{Ошибка экспорта сетки} %% подпись к рисунку
	\end{center}
\end{figure}

Для её решения достаточно удалить лишние символы:
\begin{figure}[h]
	\begin{center}
		\includegraphics[width=1\linewidth]{er.png}
		\caption{Решение ошибки экспорта сетки} %% подпись к рисунку
	\end{center}
\end{figure}


\end{document}